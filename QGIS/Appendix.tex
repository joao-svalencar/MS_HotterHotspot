\documentclass{article}
\usepackage[utf8]{inputenc}
\usepackage[T1]{fontenc}
\usepackage[brazil]{babel}
\usepackage[a4paper, total={6in,8in}]{geometry}

\begin{document}

\Large
\begin{center} 
\textbf{Hotspot getting hotter: the rise in endemism levels and habitat loss and the plight of the Cerrado savannas}
\end{center}

\normalsize
\begin{center}
João Paulo dos Santos Vieira-Alencar, Ana Paula Carmignotto, Ricardo J. Sawaya, Luis Fábio Silveira, Paula H. Valdujo and Cristiano de C. Nogueira
\end{center}

\large
\begin{center}
\textbf{Appendix 1. Step-by-step species area delimitation - lab notebook}
\end{center}
\noindent
\large{\textbf{1. Download the ottobasin shapefile.}}
\\
\normalsize
Access the website ... and download the ottobasin shapefile.
\\
\\
\large{\textbf{2. Loading shapefiles.}}
\\
\normalsize
Load the ottobasin and the study-area shapefiles in QGIS.
\\
\\
\large{\textbf{3. Selecting interest ottobasins.}}
\\
\normalsize
Use the dropdown menu and access: \textit{Vector/Research Tools/Select by location}. In the open window, select the ottobasins shapefile in the option \textit{"Select features from"} and, select the study-area shapefile in the option \textit{"by comparing to the features from"}. Then, clic \textit{run}.
\\
\\
\large{\textbf{4. Export.}}
\\
\normalsize
On the layers menu, click with the right button on the ottobasins shapefile and select the \textit{Export/Save selected features as...} option. Select the destination folder and save as a ESRI shapefile.
\\
\\
\end{document}